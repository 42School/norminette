\documentclass{42-ko}
\newcommand\qdsh{\texttt{42sh}}



%******************************************************************************%
%                                                                              %
%                                   프롤로그                                   %
%                                                                              %
%******************************************************************************%

\begin{document}
\title{The Norm}
\subtitle{Version 3}

\summary
{
    본 문서는 42에서 적용 가능한 표준(Norm)을 설명합니다.
    프로그래밍 표준은 코드를 작성할 때에 따라야하는 규칙들을 정의합니다.
    Norm은 기본적으로 이너 써클 내의 모든 C 프로젝트와
    지정된 모든 프로젝트에 적용됩니다.
}

\maketitle

\tableofcontents



%******************************************************************************%
%                                                                              %
%                                     머리말                                   %
%                                                                              %
%******************************************************************************%
\chapter{머리말}

    Norm은 파이썬으로 작성되었으며 오픈 소스입니다.
    리포지터리는 다음 주소에서 확인할 수 있습니다.
    https://github.com/42School/norminette
    풀 리퀘스트, 제안과 이슈를 환영합니다!

%******************************************************************************%
%                                                                              %
%                                    The Norm                                  %
%                                                                              %
%******************************************************************************%
\chapter{The Norm}


%******************************************************************************%
%                                   명명 규칙                                  %
%******************************************************************************%
    \section{명명}

        \begin{itemize}

            \item 구조체의 이름은 \texttt{s\_} 로 시작해야만 합니다.

            \item typedef의 이름은 \texttt{t\_} 로 시작해야만 합니다.

            \item 공용체(union)의 이름은 \texttt{u\_} 로 시작해야만 합니다.

            \item 열거형(enum)의 이름은 \texttt{e\_} 로 시작해야만 합니다.

            \item 전역 변수의 이름은 \texttt{g\_} 로 시작해야만 합니다.

            \item 변수와 함수의 이름에는 소문자, 숫자 및
            '\_' (Unix Case)만이 포함될 수 있습니다.

            \item 파일 및 디렉터리의 이름에는 소문자, 숫자 및
            '\_' (Unix Case)만이 포함될 수 있습니다.

            \item 표준 ASCII 코드표에 없는 문자는 금지됩니다.

            \item 변수, 함수 및 기타 식별자는 스네이크 케이스를 사용해야 합니다.
                대문자는 없고 각 단어는 밑줄 문자로 구분됩니다.

            \item 모든 식별자(함수, 매크로, 자료형, 변수 등)는 영어여야만 합니다.

            \item 객체(변수, 함수, 매크로, 자료형, 파일 또는 디렉터리)는
                가능한 가장 명시적이거나 가장 연상되는 이름을 가져야 합니다.

            \item 프로젝트에서 명시적으로 허용하지 않는 한
                상수(const) 및 정적(static)이 아닌 전역 변수 사용은 금지되며
                Norm 오류로 간주됩니다.

            \item 파일은 컴파일이 가능해야 합니다. 컴파일되지 않는 파일은
                Norm을 통과할 수 없을 것입니다.
        \end{itemize}
\newpage

%******************************************************************************%
%                                     서식                                     %
%******************************************************************************%
    \section{서식}

            \begin{itemize}

                \item 들여쓰기는 네 칸 크기의 탭으로 이루어져야 합니다.
                  일반적인 공백 네 칸이 아니라 진짜 탭을 말합니다.

                \item 각 함수는 함수 자체의 중괄호를 제외하고
                  최대 25줄이어야 합니다.

                \item 각 줄은 주석을 포함해 최대 80자의 열 너비를 가집니다.
                  주의: 탭 들여쓰기는 한 열로 계산하지 않으며,
                  탭이 해당되는 공백의 수 만큼으로 계산됩니다.

                \item 각 함수는 줄 바꿈으로 구분해야 합니다.
                  모든 주석과 전처리기 명령은 함수 바로 위에 있을 수 있습니다.
                  줄 바꿈은 이전 함수 다음에 와야 합니다.

                \item 한 줄에 한 명령만이 존재할 수 있습니다.

                \item 빈 줄은 공백이나 탭 들여쓰기 없이 비어 있어야 합니다.

                \item 줄은 공백이나 탭 들여쓰기로 끝날 수 없습니다.

                \item 두 개의 연속된 공백이 있을 수 없습니다.

                \item 모든 중괄호나 제어 구조 뒤는 줄바꿈으로 시작돼야 합니다.

                \item 줄의 끝이 아니라면 모든 콤마와 세미콜론 뒤에는 공백 문자가
                  따라와야 합니다.

                \item 모든 연산자나 피연산자는 하나의 공백으로 구분해야 합니다.

                \item 각 C 키워드 뒤에는 공백이 있어야만 합니다.
                  자료형 키워드(int, char, float, 등)와 sizeof는 제외됩니다.

                \item 각 변수 선언은 해당 스코프와 같은 열로 들여쓰기 되어야만 합니다.

                \item 포인터와 함께 쓰이는 별표는 변수 이름에 붙어있어야만 합니다.

                \item 한 줄에 한 개의 변수 선언만이 가능합니다.

                \item 선언과 초기화는 같은 줄에서 작성될 수 없습니다.
                  다음 경우에는 제외됩니다.
                  전역 변수(허용 될때에), 정적 변수, 그리고 상수.

                \item 선언문은 함수의 처음에 존재해야 합니다.

                \item 함수 내의 변수 선언문과 이후 함수 사이에는 빈 줄이
                  존재해야만 합니다. 다른 빈 줄은 함수 내에서 허용되지 않습니다.

                \item 다중 대입은 엄격하게 금지됩니다.

                \item 명령문이나 제어 구조 다음에 새 줄을 추가할 수도 있습니다.
                  그러기 위해서는 들여쓰기와 함께 중괄호나 대입 연산자를
                  추가해야 합니다. 연산자는 줄의 시작에 있어야만 합니다.

                \item 한 줄인 경우를 제외 하고 조건문(if, while..)에는
                  중괄호가 존재해야 합니다

                \item 함수, 선언문, 제어 구조 다음에 오는 중괄호 앞 뒤에는
                  줄바꿈이 있어야만 합니다.
            \end{itemize}

            \newpage

            일반적인 예시:
            \begin{42ccode}
int             g_global;
typedef struct  s_struct
{
    char    *my_string;
    int     i;
}               t_struct;
struct          s_other_struct;

int     main(void)
{
    int     i;
    char    c;

    return (i);
}
            \end{42ccode}
            \newpage

%******************************************************************************%
%                                 함수 매개변수                                %
%******************************************************************************%
    \section{함수}

        \begin{itemize}

            \item 한 함수에는 최대 4개의 명명된 매개변수를 가질 수 있습니다.

            \item 인자를 받지 않는 함수는 "void"라는 단어를 인자로
                명시적으로 프로토타입 돼야 합니다.

            \item 함수 프로토타입 안의 매개 변수는 명명되어야만 합니다.

            \item 각 함수는 빈 줄로 다음 함수와 구분되어야 합니다.

            \item 각 함수에서 5개를 초과하여 변수를 선언할 수 없습니다.

            \item 함수의 리턴은 괄호 사이에 있어야 합니다.

            \item 각 함수의 리턴 자료형과 함수 이름 사이에는
                한 번의 탭 들여쓰기가 있어야 합니다.

            \begin{42ccode}
int my_func(int arg1, char arg2, char *arg3)
{
    return (my_val);
}

int func2(void)
{
    return ;
}
            \end{42ccode}

        \end{itemize}
        \newpage


%******************************************************************************%
%                  자료형, 구조체, 열거형(enum)과 공용체(union)                %
%******************************************************************************%
    \section{자료형, 구조체, 열거형(enum)과 공용체(union)}

        \begin{itemize}

            \item 구조체, 열거형, 공용체를 선언 할 때는 탭 들여쓰기를 넣습니다.

            \item 구조체, 열거형, 공용체의 변수를 선언할 때에는
                자료형 안에 공백 문자 하나를 넣습니다.

            \item 구조체, 열거형, 공용체를 typedef와 함께 선언할 때에는
                모든 들여쓰기 규칙이 적용됩니다.

            \item typedef 이름은 탭이 앞에 있어야만 합니다.

            \item 모든 구조체의 이름은 해당 스코프의 같은 열에 들여쓰기 해야만 합니다.

            \item .c 파일 내에서 구조체를 선언할 수 없습니다.

        \end{itemize}
        \newpage


%******************************************************************************%
%                                      헤더                                    %
%******************************************************************************%
    \section{헤더}

        \begin{itemize}

            \item 헤더파일에서 허용되는 것들:
                헤더 인클루드(시스템 헤더 또는 유저 헤더),
                선언문, defines, 프로토타입과 매크로.

            \item 모든 인클루드는 파일의 시작에 작성되어야 합니다.

            \item C 파일을 포함할 수 없습니다.

            \item 헤더 파일은 중복 인클루드를 방지해야만 합니다.
                만약 파일 이름이 ft\_foo.h라면 인클루드 가드 매크로 이름은
                FT\_FOO_H 가 되어야 합니다.

            \item 사용하지 않은 헤더의 인클루드는 금지됩니다.

            \item .c / .h 파일의 모든 헤더 인클루드는 정당한 이유가 있어야만 합니다.

        \end{itemize}

        \begin{42ccode}
#ifndef FT_HEADER_H
# define FT_HEADER_H
# include <stdlib.h>
# include <stdio.h>
# define FOO "bar"

int		g_variable;
struct	s_struct;

#endif
        \end{42ccode}
        \newpage

%******************************************************************************%
%                              매크로와 전처리기                               %
%******************************************************************************%
    \section{매크로와 전처리기}

        \begin{itemize}

            \item 매크로 상수(또는 \#define)는 리터럴이나
                상숫값에만 사용 가능합니다.
            \item Norm을 우회하거나 코드 가독성을 낮추는 모든 \#define은 금지됩니다.
                이 부분은 사람에 의해 검사되어야 합니다.
            \item 표준 라이브러리의 매크로는 프로젝트에서
                사용이 허가되었을 경우에만 사용 가능합니다.
            \item 여러 줄에 걸친 매크로는 금지됩니다.
            \item 매크로 이름은 모두 대문자여야만 합니다.
            \item \#if, \#ifdef, \#ifndef 다음 문자들은 들여쓰기 해야만 합니다.

        \end{itemize}
        \newpage


%******************************************************************************%
%                                  금지 사항!                                  %
%******************************************************************************%
    \section{금지 사항!}

        \begin{itemize}

            \item 다음 구문은 사용이 금지됩니다:

                \begin{itemize}

                    \item for
                    \item do...while
                    \item switch
                    \item case
                    \item goto

                \end{itemize}

            \item 다음과 같은 삼항 연산자 `?'.

            \item VLA - 가변 길이 배열.

            \item 자료형을 명시하지 않은 변수 선언

        \end{itemize}
        \begin{42ccode}
    int main(int argc, char **argv)
    {
        int     i;
        char    string[argc]; // 가변 길이 배열

        i = argc > 5 ? 0 : 1 // 삼항 연산자
    }
        \end{42ccode}
        \newpage

%******************************************************************************%
%                                   주석                                       %
%******************************************************************************%
    \section{주석}

        \begin{itemize}

            \item 주석은 함수 내부에 있을 수 없습니다. 
                주석은 줄 끝에 있거나 별개의 줄에 있어야만 합니다.

            \item 주석은 영어여야만 합니다. 그리고 유용해야만 합니다.

            \item 주석은 "쓰레기 같은" 함수를 정당화할 수 없습니다.

        \end{itemize}
        \newpage


%******************************************************************************%
%                                    파일                                      %
%******************************************************************************%
    \section{파일}

        \begin{itemize}

            \item .c 파일을 인클루드할 수 없습니다.

            \item 하나의 .c 파일에 함수를 5개보다 많이 정의할 수 없습니다.

        \end{itemize}
        \newpage


%******************************************************************************%
%                                   Makefile                                   %
%******************************************************************************%
    \section{Makefile}

            Makefile은 Norm에서 확인하지 않으며, 반드시
              학생이 평가 중에 확인해야만 합니다.
            \begin{itemize}

                \item 다음 규칙은 필수적입니다.
                  \$(NAME), clean, fclean, re and all

                \item Makefile이 리링크(relink)되면, 프로젝트는
                  작동하지 않는 것으로 간주됩니다.

                \item 실행 파일이 여러 개인 프로젝트의 경우, 위의 규칙 이외에도
                  컴파일된 각각의 실행 파일에 대한 특정 규칙 뿐만 아니라
                  실행 파일들을 모두 컴파일하는 규칙이 있어야만 합니다.

                  \item 비-시스템 라이브러리(예: \texttt{libft})에서
                    함수를 호출하는 프로젝트의 경우 Makefile은
                    반드시 이 라이브러리를 자동으로 컴파일해야만 합니다.

                  \item 프로젝트를 컴파일하기 위해 필요한 모든 소스파일들은 
                    Makefile에 반드시 명시해야만 합니다.

            \end{itemize}



\end{document}
%******************************************************************************%
