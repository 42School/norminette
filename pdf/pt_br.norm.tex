\documentclass{42-pt}



%******************************************************************************%
%                                                                              %
%                               Prologue                                       %
%                                                                              %
%******************************************************************************%

\begin{document}
\title{A Norma}
\subtitle{Versão 4}

\summary
{
    Este documento descreve o padrão aplicável (Norma) na 42. Um padrão de
	programação define um conjunto de regras a seguir ao escrever um código.
    A norma aplica-se a todos os projetos C dentro do círculo interno por padrão, e
    para qualquer projeto onde é especificado.
}

\maketitle

\tableofcontents



%******************************************************************************%
%                                                                              %
%                                 Introdução                                   %
%                                                                              %
%******************************************************************************%
\chapter{Introdução}

    A \texttt{norminette} é em Python e código aberto.\\
    Seu repositório está disponível em https://github.com/42school/norminette. \\
    Pull requests, sugestões e indicação de bugs são bem-vindos!


%******************************************************************************%
%                                                                              %
%                                   Explicações Pedago                         %
%                                                                              %
%******************************************************************************%
    \chapter{Por quê ?}

    A Norma foi cuidadosamente elaborada para suprir diversas necessidades
    pedagógicas. Aqui estão alguns dos motivos mais importantes por trás das
    escolhas abaixo:
    \begin{itemize}

	\item Sequenciamento: programar implica dividir uma tarefa grande e
	  complexa em uma série de instruções elementares. Todas essas instruções
	  vão ser executadas em sequência: uma após a outra. Um iniciante, ao
      começar a criar software, precisa de uma arquitetura simples e
	  clara para seu projeto, tendo o entendimento completo de todas as
      instruções individuais e da exata ordem de execução. Sintaxes de
      linguagens crípticas que aparentam executar múltiplas
	  instruções ao mesmo tempo são confusas, funções que buscam abordar
      múltiplas tarefas misturadas na mesma porção de código são fontes
      de erros. \\
	  A Norma pede que você escreva trechos simples de código cujas tarefas
	  possam ser entendidas e verificadas facilmente, em que a sequência de
      execução das instruções não deixa dúvidas. Por este motivo que há o
      limite máximo de 25 linhas por função, e o porquê de \texttt{for},
      \texttt{do .. while}, ou ternários serem proibidos.


    \item Estética: enquanto se relaciona com seus colegas durante o processo
      natural de aprendizado entre pares, e também durante as avaliações
      entre pares, você não quer gastar tempo decifrando o código deles,
      mas falar diretamente sobre a lógica daquele trecho de código.\\ A
      Norma pede por uma estética específica, provendo instruções para
      nomear funções e variáveis, indentação, utilização das chaves,
      tabulações e espaços em diversos lugares... . Isso vai permitir que
      você olhe brevemente para o código de outros e o ache familiar, podendo
      ir direto ao assunto, ao invés de gastar tempo lendo o código antes
      de entendê-lo. A Norma também se caracteriza como uma marca
      registrada. Como parte da comunidade 42, você vai poder reconhecer
      código escrito por outro cadete ou alumni da 42 quando estiver no
      mercado de trabalho.
    

    \item Visão de longo prazo: esforçar-se para escrever um código compreensível
      é a melhor maneira de administrá-lo. Toda vez que alguém, incluindo
      você, precisar consertar um bug ou adicionar uma nova
      funcionalidade, não será necessário gastar tempo tentando entender
      o funcionamento se você escreveu seu código da maneira correta. Isso
      vai evitar situações em que trechos de código deixam de ser
      atualizados apenas por tomarem tempo, o que vai fazer a diferença
      ao falarmos sobre ter um produto bem sucedido no mercado. Quanto
      mais cedo aprender, melhor.


    \item Referências: você pode pensar que algumas, ou todas, as regras
      inclusas na Norma são arbitrárias, mas nós pensamos cuidadosamente e
      pesquisamos como elaborá-la. Nós encorajamos fortemente que você
      pesquise o porquê de funções precisarem ser curtas e possuir apenas
      uma tarefa, o porquê de nomes de variáveis precisarem ser
      compreensíveis, o porquê de linhas não poderem extrapolar o limite
      de 80 colunas de largura, o porquê de uma função não poder receber
      vários parâmetros, o porquê de comentários serem úteis, etc, etc,
      etc ...


    \end{itemize}


\newpage

%******************************************************************************%
%                                                                              %
%                                  A Norma                                     %
%                                                                              %
%******************************************************************************%
\chapter{A Norma}


%******************************************************************************%
%                            Convenções de nomeação                            %
%******************************************************************************%
    \section{Denominação}

        \begin{itemize}

            \item O nome de um struct deve começar por
                \texttt{s\_}.

            \item O nome de um typedef deve começar por
                \texttt{t\_}.

            \item O nome de um union deve começar por \texttt{u\_}.

            \item O nome de um enum deve começar por \texttt{e\_}.

            \item O nome de uma variável global deve começar por \texttt{g\_}.

            \item Os nomes de variáveis e funções só podem conter letras minúsculas,
             dígitos e '\_' (Unix Case).

            \item Os nomes de arquivos e diretórios só podem conter minúsculas, dígitos e
                '\_' (Unix Case).

            \item Caracteres que não fazem parte 
                da tabela ASCII padrão é proibida.

            \item Variáveis, funções e qualquer outro identificador deve usar
                snake case. Sem letras maiúsculas, e cada palavra separada por um
                sublinhado.

            \item Todos os identificadores (funções, macros, tipos,
                variáveis, etc.) devem estar em inglês.

            \item Objetos (variáveis, funções, macros, tipos,
                arquivos ou diretórios) devem ter os
                nomes mais explícitos ou mais mnemônicos possíveis.

            \item Usar variáveis globais é proibido e considerado um erro de norma, há menos que seja explicitamente permitido o uso no projeto.

            \item O arquivo deve compilar. Um arquivo que não compila não é esperado
                que passe na norma.
        \end{itemize}
\newpage

%******************************************************************************%
%                                 Formatação                                   %
%******************************************************************************%
    \section{Formatação}

            \begin{itemize}

                \item Você deve indentar seu código com tabulação de 4 espaços.
                  Isto não é o mesmo que 4
                  espaços, estamos falando de tabulações reais aqui.

                \item Cada função deve ter no máximo 25 linhas, não
                  contando suas próprias chaves '\{\}'.

                \item Cada linha deve ter no máximo 80 colunas de largura,
                  comentários incluídos. Aviso: uma tabulação não conta
                  como uma coluna, mas como o número de espaços que
                  representa.

                \item Cada função deve ser separada por uma nova linha. Qualquer
                  comentário ou instrução de pré-processador
                  pode estar logo acima da função. A nova linha é após a função anterior.

                \item Apenas uma instrução por linha.

                \item Uma linha vazia deve estar vazia: sem espaços ou tabulações.

                \item Uma linha nunca pode terminar com espaços ou tabulações.

                \item Você nunca pode ter dois espaços consecutivos.

                \item Você precisa iniciar uma nova linha após cada chave '\{\}'
                  ou final da estrutura de controle.

                \item A menos que seja o fim de uma linha, cada vírgula ou ponto-e-vírgula
                  deve ser seguido por um espaço.

                \item Cada operador ou operando deve ser separado por um
                 - e apenas um - espaço.

                \item Cada palavra-chave do C deve ser seguida por um espaço, exceto
                  Palavras-chave para tipos (como int, char, float, etc.),
                  bem como sizeof.

                \item Cada declaração de variável deve ser indentada na mesmo
                  coluna para seu escopo.

                \item Os asteriscos que vão com ponteiros devem estar juntos aos
                  nomes das variáveis.

                \item Uma única declaração de variável por linha.

                \item Declaração e uma inicialização não podem estar
                  na mesma linha, exceto para variáveis globais (quando permitido),
                  variáveis estáticas e constantes.

                \item Declarações devem estar no início de uma função.

                \item Em uma função, você deve colocar uma linha vazia entre as
                    declarações de variáveis e o restante da função.
                    Nenhuma outra linha vazia é permitida em uma função.

                \item Atribuições múltiplas são estritamente proibidas.

                \item Você pode adicionar uma nova linha após uma instrução ou
                  estrutura de controle, mas você terá que adicionar um
                  indentação com chaves ou operador de atribuição.
                  Os operadores devem estar no início de uma linha.

                \item Estruturas de controle (if, while ...) devem ter chaves,
                a menos que contenham uma única linha.
	\item Funções seguidas de chaves, declaradores ou estruturas de controle devem ser precedidas e seguidas de uma nova linha.

            \end{itemize}

            \newpage

            General example:
            \begin{42ccode}
int             g_global;
typedef struct  s_struct
{
    char    *my_string;
    int     i;
}               t_struct;
struct          s_other_struct;

int     main(void)
{
    int     i;
    char    c;

    return (i);
}
            \end{42ccode}
            \newpage

%******************************************************************************%
%                              Parâmetros de função                             %
%******************************************************************************%
    \section{Funções}

        \begin{itemize}

            \item Uma função pode ter até 4 parâmetros definidos no máximo.

            \item Uma função que não tem argumentos deve ser
                explicitamente prototipada com a palavra "void" como o
                argumento.

            \item Parâmetros em protótipos de funções devem ser nomeados.

            \item Cada função deve ser separada da próxima por
                uma linha vazia.

            \item Você não pode declarar mais de 5 variáveis por função.

            \item O retorno de uma função deve estar entre parênteses. 

            \item Cada função deve ter uma tabulação única entre seu
                tipo de retorno e seu nome.

            \begin{42ccode}
int my_func(int arg1, char arg2, char *arg3)
{
    return (my_val);
}

int func2(void)
{
    return ;
}
            \end{42ccode}

        \end{itemize}
        \newpage


%******************************************************************************%
%                        Typedef, struct, enum e union                       %
%******************************************************************************%
    \section{Typedef, struct, enum e union}

        \begin{itemize}

            \item Adicione uma tabulação ao declarar um struct, enum ou union.

            \item Ao declarar uma variável do tipo struct, enum ou union,
                adicione um único espaço no tipo.

            \item Ao declarar um struct, union ou enum com um typedef,
                todas as regras de indentação aplicam-se.

            \item O nome de um typedef deve ser precedido por uma tabulação.

            \item Você deve recuar todos os nomes de estruturas na mesma coluna
                para o escopo deles.

            \item Você não pode declarar uma estrutura em um arquivo .c.

        \end{itemize}
        \newpage


%******************************************************************************%
%                                   Headers                                    %
%******************************************************************************%
    \section{Cabeçalhos}

        \begin{itemize}

            \item As coisas permitidas em arquivos de cabeçalho são:
                Inclusões de cabeçalho (sistema ou não), declarações, definições,
                protótipos e macros.

            \item Todas inclusões devem estar no início do arquivo.

            \item Você não pode incluir um arquivo C.

            \item Os arquivos de cabeçalho devem ser protegidos contra inclusões
             duplas. Se o arquivo é \texttt{ft\_foo.h}, a macro que o acompanhará
             é \texttt{FT\_FOO\_H}.

            \item Inclusões de cabeçalho não utilizadas (.h) são proibidas.

            \item Todas as inclusões de cabeçalho devem ser justificadas em um arquivo .c
                bem como em um arquivo .h.

        \end{itemize}

        \begin{42ccode}
#ifndef FT_HEADER_H
# define FT_HEADER_H
# include <stdlib.h>
# include <stdio.h>
# define FOO "bar"

int g_variable;
struct s_struct;

#endif
        \end{42ccode}
        \newpage

%******************************************************************************%
%                                 O Header da 42                               %
%******************************************************************************%

   \section{O Header da 42 - vulgo começar um arquivo com estilo}

        \begin{itemize}

        \item Todo arquivo .c e .h deve, imediatamente, começar com o header da 42:
          um comentário multilinha com um formato específico, incluindo informações
          úteis. O header padrão se encontra naturalmente disponível nos computadores
          dos clusters para diversos editores de texto (emacs: usando \texttt{C-c C-h},
          vim: usando \texttt{:Stdheader} ou \texttt{F1}, etc...)

        \item O header da 42 deve conter informações atualizadas, incluindo o
          criador com login e email, a data de criação e a data da atualização
          mais recente. Cada vez que o arquivo é salvo em disco, a informação deve ser atualizada automaticamente.

        \end{itemize}
        \newpage
        
                
%******************************************************************************%
%                           Macros e pré-processadores                         %
%******************************************************************************%
    \section{Macros e pré-processadores}

        \begin{itemize}

            \item Constantes do pré-processador (ou \#define) que você criar devem ser usadas
                apenas para valores literais e constantes.
            \item Todos \#define criados para ignorar a norma e / ou ofuscação de
                código são proibidos. Esta parte deve ser verificada por um humano.
            \item Você pode usar macros disponíveis em bibliotecas padrão, apenas
                se estas são permitidas no escopo do projeto.
            \item Macros multilinhas são proibidas.
            \item Nomes de macro devem ser todos maiúsculos (uppercase).
            \item Você deve recuar caracteres que seguirem \#if, \#ifdef
                or \#ifndef.
	\item Instruções de pré-processamento são proibidas fora do escopo global.
        \end{itemize}
        \newpage


%******************************************************************************%
%                              Coisas proibidas!                               %
%******************************************************************************%
    \section{Coisas proibidas!}

        \begin{itemize}

            \item Você não tem permissão para usar:

                \begin{itemize}

                    \item for
                    \item do...while
                    \item switch
                    \item case
                    \item goto

                \end{itemize}

            \item Operadores ternários como `?'.

            \item VLAs - Matrizes de comprimento variável.

            \item Tipo implícito em declarações variáveis.

        \end{itemize}
        \begin{42ccode}
    int main(int argc, char **argv)
    {
        int     i;
        char    string[argc]; // This is a VLA

        i = argc > 5 ? 0 : 1 // Ternary
    }
        \end{42ccode}
        \newpage

%******************************************************************************%
%                                   Comments                                   %
%******************************************************************************%
    \section{Comentários}

        \begin{itemize}

            \item Os comentários não podem estar dentro do corpo das funções.
                Os comentários devem estar no final de uma linha, ou em sua própria linha

            \item Seus comentários devem estar em inglês. E eles devem ser
                úteis.

            \item Um comentário não pode ser usado para justificar declarações ou uma função 
				mal feita.

        \end{itemize}
	\warn{
		Uma declaração ou função mal feita normalmente vem com nomes não inteligíveis como f1, f2, etc. para as funções e, a, b, i, etc. para declarações.
		Uma função cujo objetivo é evitar ou burlar a norma, sem um único propósito lógico, também é considerado uma função mal feita.
		Por favor lembre se que é desejável ter funções limpas e legíveis que realizem uma clara e simples tarefa cada. Evite qualquer técnica de ofuscação do código como \emph{one-liner}..
        \newpage


%******************************************************************************%
%                                    Files                                     %
%******************************************************************************%
    \section{Arquivos}

        \begin{itemize}

            \item Você não pode incluir um arquivo .c.

            \item Você não pode ter mais de 5 definições de função em um arquivo .c.

        \end{itemize}
        \newpage


%******************************************************************************%
%                                   Makefile                                   %
%******************************************************************************%
    \section{Makefile}

            Makefiles não são verificados pela norma e devem ser verificados durante a avaliação pelo estudante.
            \begin{itemize}

                \item AS regras \$(NAME), clean, fclean, re and all
                  são obrigatórias.

                \item Se o makefile fizer relink, o projeto será considerado
                  não funcional.

                \item No caso de um projeto multibinário, além das
                  regras acima, você deve ter uma regra que compila
                  ambos os binários, bem como uma regra específica para cada
                  binário compilado.

                \item No caso de um projeto que chama uma função de uma biblioteca não-sistema
                  (por exemplo: \texttt{libft}), seu makefile deve compilar
                  esta biblioteca automaticamente.

                \item Todos os arquivos de origem que você precisa compilar seu projeto deve
                    ser explicitamente nomeado em seu makefile.

            \end{itemize}



\end{document}
%******************************************************************************%
