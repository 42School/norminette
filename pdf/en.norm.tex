\documentclass{42-en}



%******************************************************************************%
%                                                                              %
%                               Prologue                                       %
%                                                                              %
%******************************************************************************%

\begin{document}
\title{The Norm}
\subtitle{Version 3}

\summary
{
    This document describes the applicable standard (Norm) at 42. A
    programming standard defines a set of rules to follow when writing code.
    The Norm applies to all C projects within the Common Core by default, and
    to any project where it's specified.
}

\maketitle

\tableofcontents



%******************************************************************************%
%                                                                              %
%                                 Foreword                                     %
%                                                                              %
%******************************************************************************%
\chapter{Foreword}

    The \texttt{norminette} is in python and open source. \\
    Its repository is available at https://github.com/42School/norminette.\\
    Pull requests, suggestions and issues are welcome!

   \newpage


%******************************************************************************%
%
%                                   Pedago explanations                        %
%
%******************************************************************************%
    \chapter{Why ?}

    The Norm has been carefully crafted to fulfill many pedagogical needs. Here
    are the most important reasons for all the choices below:
    \begin{itemize}

    \item Sequencing: coding implies splitting a big and complex task into a long series
      of elementary instructions. All these instructions will be executed in sequence:
      one after another. A beginner that starts creating software needs a simple and clear
      architecture for their project, with a full understanding of all individual instructions
      and the precise order of execution. Cryptic language syntaxes that do multiple
      instructions apparently at the same time are confusing, functions that try to address
      multiple tasks mixed in the same portion of code are source of errors.\\
      The Norm asks you to create simple pieces of code, where the unique task of each piece
      can be clearly understood and verified, and where the sequence of all the executed
      instructions leaves no doubt. That's why we ask for 25 lines maximum in functions, also why
      \texttt{for}, \texttt{do .. while}, or ternaries are forbidden.

    \item Look and Feel: while exchanging with your friends and workmates during the
      normal peer-learning process, and also during the peer-evaluations, you do not
      want to spend time to decrypt their code, but directly talk about the
      logic of the piece of code.\\
      The Norm asks you to use a specific look and feel, providing instructions for the naming
      of the functions and variables, indentation, brace rules, tab and spaces at many places... .
      This will allow you to smoothly have a look at other's codes that will look familiar,
      and get directly to the point instead of spending time to read the code before understanding it.
      The Norm also comes as a trademark. As part of the 42 community, you will be able to
      recognize code written by another 42 student or alumni when you'll be in the labor market.
    
    \item Long-term vision: making the effort to write understandable code is the
      best way to maintain it. Each time that someone else, including you, has to fix a bug
      or add a new feature they won't have to lose their precious time trying to figure out
      what it does if previously you did things in the right way. This will avoid situations
      where pieces of code stop being maintained just because it is time-consuming, and that
      can make the difference when we talk about having a successful product in the market.
      The sooner you learn to do so, the better.

    \item References: you may think that some, or all, the rules included on the Norm are
      arbitrary, but we actually thought and read about what to do and how to do it. We highly
      encourage you to Google why the functions should be short and just do one thing, why the
      name of the variables should make sense, why lines shouldn't be longer than 80 columns wide,
      why a function should not take many parameters, why comments should be useful, etc, etc, etc...

    \end{itemize}


\newpage

%******************************************************************************%
%                                                                              %
%                                The Norm                                      %
%                                                                              %
%******************************************************************************%
\chapter{The Norm}


%******************************************************************************%
%                             Naming conventions                               %
%******************************************************************************%
    \section{Denomination}

        \begin{itemize}

            \item A structure's name must start by
                \texttt{s\_}.

            \item A typedef's name must start by
                \texttt{t\_}.

            \item A union's name must start by \texttt{u\_}.

            \item An enum's name must start by \texttt{e\_}.

            \item A global's name must start by \texttt{g\_}.

            \item Variables and functions names can only contain lowercases, digits and
                '\_' (Unix Case).

            \item Files and directories names can only contain lowercases, digits and
                '\_' (Unix Case).

            \item Characters that aren't part of the standard
                ASCII table are forbidden.

            \item Variables, functions, and any other identifier must use
                snake case. No capital letters, and each word separated by an 
                underscore.

            \item All identifiers (functions, macros, types,
                variables, etc.) must be in English.

            \item Objects (variables, functions, macros, types,
                files or directories) must have the most
                explicit or most mnemonic names as possible.

            \item Declaring global variables that are not marked const and static is 
            forbidden and is considered a norm error, unless the project explicitly allows them.

            \item The file must compile. A file that doesn't compile isn't expected
                to pass the Norm.
        \end{itemize}
\newpage

%******************************************************************************%
%                                 Formatting                                   %
%******************************************************************************%
    \section{Formatting}

            \begin{itemize}

                \item You must indent your code with 4-space
                  tabulations. This is not the same as 4 average
                  spaces, we're talking about real tabulations here.

                \item Each function must be maximum 25 lines, not
                  counting the function's own curly brackets.

                \item Each line must be at most 80 columns wide, comments
                  included. Warning: a tabulation doesn't count
                  as a column, but as the number of spaces it
                  represents.

                \item Each function must be separated by a newline. Any comment or preprocessor instruction
                    can be right above the function. The newline is after the previous function.

                \item One instruction per line.

                \item An empty line must be empty: no spaces or tabulations.

                \item A line can never end with spaces or tabulations.

                \item You can never have two consecutive spaces.

                \item You need to start a new line after each curly bracket
                  or end of control structure.

                \item Unless it's the end of a line, each comma or semi-colon
                  must be followed by a space.

                \item Each operator or operand must be separated by one
                 - and only one - space.

                \item Each C keyword must be followed by a space, except for
                  keywords for types (such as int, char, float, etc.),
                  as well as sizeof.

                \item Each variable declaration must be indented on the same
                  column for its scope.

                \item The asterisks that go with pointers must be stuck to
                  variable names.

                \item One single variable declaration per line.

                \item Declaration and an initialisation cannot be
                  on the same line, except for global variables (when allowed),
                  static variables, and constants.

                \item Declarations must be at the beginning of a function.

                \item In a function, you must place an empty line between 
                    variable declarations and the remaining of the function.
                    No other empty lines are allowed in a function.

                \item Multiple assignments are strictly forbidden.

                \item You may add a new line after an instruction or
                  control structure, but you'll have to add an
                  indentation with brackets or assignment operator.
                  Operators must be at the beginning of a line.

                \item Control structures (if, while..) must have braces, unless they contain a single 
                    line.

                \item Braces following functions, declarators or control structures must be preceded and followed by a newline.

            \end{itemize}

            \newpage

            General example:
            \begin{42ccode}
int             g_global;
typedef struct  s_struct
{
    char    *my_string;
    int     i;
}               t_struct;
struct          s_other_struct;

int     main(void)
{
    int     i;
    char    c;

    return (i);
}
            \end{42ccode}
            \newpage

%******************************************************************************%
%                              Function parameters                             %
%******************************************************************************%
    \section{Functions}

        \begin{itemize}

            \item A function can take 4 named parameters maximum.

            \item A function that doesn't take arguments must be
                explicitly prototyped with the word "void" as the
                argument.

            \item Parameters in functions' prototypes must be named.

            \item Each function must be separated from the next by
                an empty line.

            \item You can't declare more than 5 variables per function.

            \item Return of a function has to be between parenthesis. 

            \item Each function must have a single tabulation between its
                return type and its name.

            \begin{42ccode}
int my_func(int arg1, char arg2, char *arg3)
{
    return (my_val);
}

int func2(void)
{
    return ;
}
            \end{42ccode}

        \end{itemize}
        \newpage


%******************************************************************************%
%                        Typedef, struct, enum and union                       %
%******************************************************************************%
    \section{Typedef, struct, enum and union}

        \begin{itemize}

            \item Add a tabulation when declaring a struct, enum or union.

            \item When declaring a variable of type struct, enum or union,
                add a single space in the type.

            \item When declaring a struct, union or enum with a typedef,
                all indentation rules apply.

            \item Typedef name must be preceded by a tab.

            \item You must indent all structures' names on the same column for their scope.

            \item You cannot declare a structure in a .c file.

        \end{itemize}
        \newpage


%******************************************************************************%
%                                   Headers                                    %
%******************************************************************************%
    \section{Headers - a.k.a include files}

        \begin{itemize}

            \item The things allowed in header files are:
                header inclusions (system or not), declarations, defines,
                prototypes and macros.

            \item All includes must be at the beginning of the file.

            \item You cannot include a C file.

            \item Header files must be protected from double inclusions. If the file is
            \texttt{ft\_foo.h}, its bystander macro is \texttt{FT\_FOO\_H}.

            \item Unused header inclusions (.h) are forbidden.

            \item All header inclusions must be justified in a .c file
                as well as in a .h file.

        \end{itemize}

        \begin{42ccode}
#ifndef FT_HEADER_H
# define FT_HEADER_H
# include <stdlib.h>
# include <stdio.h>
# define FOO "bar"

int		g_variable;
struct	s_struct;

#endif
        \end{42ccode}
        \newpage


%******************************************************************************%
%                                 The 42 header                                %
%******************************************************************************%

   \section{The 42 header - a.k.a start a file with style}

        \begin{itemize}

        \item Every .c and .h file must immediately begin with the standard 42 header :
          a multi-line comment with a special format including useful informations. The
          standard header is naturally available on computers in clusters for various
          text editors (emacs : using \texttt{C-c C-h}, vim using \texttt{:Stdheader} or
          \texttt{F1}, etc...).

        \item The 42 header must contain several informations up-to-date, including the
          creator with login and email, the date of creation, the login and date of the
          last update. Each time the file is saved on disk, the information should be
          automatically updated.

        \end{itemize}
        \newpage
        
                
%******************************************************************************%
%                           Macros and Pre-processors                          %
%******************************************************************************%
    \section{Macros and Pre-processors}

        \begin{itemize}

            \item Preprocessor constants (or \#define) you create must be used
                only for literal and constant values.
            \item All \#define created to bypass the norm and/or obfuscate
                code are forbidden. This part must be checked by a human.
            \item You can use macros available in standard libraries, only
                if the latter are allowed in the scope of the given project.
            \item Multiline macros are forbidden.
            \item Macro names must be all uppercase.
            \item You must indent characters following \#if, \#ifdef
                or \#ifndef.

        \end{itemize}
        \newpage


%******************************************************************************%
%                              Forbidden stuff!                                %
%******************************************************************************%
    \section{Forbidden stuff!}

        \begin{itemize}

            \item You're not allowed to use:

                \begin{itemize}

                    \item for
                    \item do...while
                    \item switch
                    \item case
                    \item goto

                \end{itemize}

            \item Ternary operators such as `?'.

            \item VLAs - Variable Length Arrays.

            \item Implicit type in variable declarations

        \end{itemize}
        \begin{42ccode}
    int main(int argc, char **argv)
    {
        int     i;
        char    string[argc]; // This is a VLA

        i = argc > 5 ? 0 : 1 // Ternary
    }
        \end{42ccode}
        \newpage

%******************************************************************************%
%                                   Comments                                   %
%******************************************************************************%
    \section{Comments}

        \begin{itemize}

            \item Comments cannot be inside functions' bodies.
                Comments must be at the end of a line, or on their own line

            \item Your comments must be in English. And they must be
                useful.

            \item A comment cannot justify a "bastard" function.

        \end{itemize}
        \newpage


%******************************************************************************%
%                                    Files                                     %
%******************************************************************************%
    \section{Files}

        \begin{itemize}

            \item You cannot include a .c file.

            \item You cannot have more than 5 function-definitions in a .c file.

        \end{itemize}
        \newpage


%******************************************************************************%
%                                   Makefile                                   %
%******************************************************************************%
    \section{Makefile}

            Makefiles aren't checked by the Norm, and must be checked during evaluation by 
            the student.
            \begin{itemize}

                \item The \$(NAME), clean, fclean, re and all
                  rules are mandatory.

                \item If the makefile relinks, the project will be considered
                  non-functional.

                \item In the case of a multibinary project, in addition to
                  the above rules, you must have a rule that compiles
                  both binaries as well as a specific rule for each
                  binary compiled.

                  \item In the case of a project that calls a function from a non-system library
                  (e.g.: \texttt{libft}), your makefile must compile
                  this library automatically.

                  \item All source files you need to compile your project must
                    be explicitly named in your Makefile.

            \end{itemize}


\end{document}
%******************************************************************************%
