\documentclass{42-en}
\newcommand\qdsh{\texttt{42sh}}



%******************************************************************************%
%                                                                              %
%                               Prologue                                       %
%                                                                              %
%******************************************************************************%

\begin{document}
\title{The Norm}
\subtitle{Version 3}

\summary
{
    This document describes the applicable standard (Norm) at 42. A
    programming standard defines a set of rules to follow when writing code.
    The Norm applies to all C projects within the Inner Circle by default, and
    to any project where it's specified.
}

\maketitle

\tableofcontents



%******************************************************************************%
%                                                                              %
%                                 Foreword                                     %
%                                                                              %
%******************************************************************************%
\chapter{Foreword}

    The Norm is in python and open source. \\
    Its repository is available at https://github.com/42School/norminette.\\
    Pull requests, suggestions and issues are welcome!

%******************************************************************************%
%                                                                              %
%                                The Norm                                      %
%                                                                              %
%******************************************************************************%
\chapter{The Norm}


%******************************************************************************%
%                             Naming conventions                               %
%******************************************************************************%
    \section{Denomination}

        \begin{itemize}

            \item A structure's name must start by
                \texttt{s\_}.

            \item A typedef's name must start by
                \texttt{t\_}.

            \item A union's name must start by \texttt{u\_}.

            \item An enum's name must start by \texttt{e\_}.

            \item A global's name must start by \texttt{g\_}.

            \item Variables and functions names can only contain lowercases, digits and
                '\_' (Unix Case).

            \item Files and directories names can only contain lowercases, digits and
                '\_' (Unix Case).

            \item Characters that aren't part of the standard
                ASCII table are forbidden.

            \item Variables, functions, and any other identifier must use
                snake case. No capital letters, and each word separated by an 
                underscore.

            \item All identifiers (functions, macros, types,
                variables, etc.) must be in English.

            \item Objects (variables, functions, macros, types,
                files or directories) must have the most
                explicit or most mnemonic names as possible.

            \item Using a global variable in a project where it's not
                explicitly allowed is a norm error, except where it's
                mandatory (signal handling for example).

            \item The file must compile. A file that doesn't compile isn't expected
                to pass the Norm.
        \end{itemize}
\newpage

%******************************************************************************%
%                                 Formatting                                   %
%******************************************************************************%
    \section{Formatting}

            \begin{itemize}

                \item You must indent your code with 4-space
                  tabulations. Indentation must be aligned with tab stops.

                \item Each function must be maximum 25 lines, not
                  counting the function's own curly brackets.

                \item Each line must be at most 80 columns wide, comments
                  included. Warning: a tabulation doesn't count
                  as a column, but as the number of spaces it
                  represents.

                \item Each function must be separated by a newline. Any comment or preprocessor instruction
                    can be right above the function. The newline is after the previous function.

                \item One instruction per line.

                \item An empty line must be empty: no spaces or tabulations.

                \item A line can never end with spaces or tabulations.

                \item You can never have two consecutive spaces outside of tabulations.

                \item You need to start a new line after each curly bracket
                  or end of control structure.

                \item Unless it's the end of a line, each comma or semi-colon
                  must be followed by a space.

                \item Each operator or operand must be separated by one
                 - and only one - space.

                \item Each C keyword must be followed by a space, except for
                  keywords for types (such as int, char, float, etc.),
                  as well as sizeof.

                \item Each variable declaration must be indented on the same
                  column for its scope.

                \item The asterisks that go with pointers must be stuck to
                  variable names.

                \item One single variable declaration per line.

                \item Declaration and an initialisation cannot be
                  on the same line, except for global variables (when allowed),
                  static variables, and constants.

                \item Declarations must be at the beginning of a function.

                \item In a function, you must place an empty line between 
                    variable declarations and the remaining of the function.
                    No other empty lines are allowed in a function.

                \item Multiple assignments are strictly forbidden.

                \item You may add a new line after an instruction or
                  control structure, but you'll have to add an
                  indentation with brackets or affectation operator.
                  Operators must be at the beginning of a line.

                \item Control structures (if, while..) must have braces, unless they contain a single 
                    line or a single condition.

            \end{itemize}

            \newpage

            General example:
            \begin{42ccode}
int             g_global;
typedef struct  s_struct
{
    char    *my_string;
    int     i;
}               t_struct;
struct          s_other_struct;

int     main(void)
{
    int     i;
    char    c;

    return (i);
}
            \end{42ccode}
            \newpage

%******************************************************************************%
%                              Function parameters                             %
%******************************************************************************%
    \section{Functions}

        \begin{itemize}

            \item A function can take 4 named parameters maximum.

            \item A function that doesn't take arguments must be
                explicitely prototyped with the word "void" as the
                argument.

            \item Parameters in functions' prototypes must be named.

            \item Each function must be separated from the next by
                an empty line.

            \item You can't declare more than 5 variables per function.

            \item Return of a function has to be between parenthesis. 

            \item Each function must have a single tabulation between its
                return type and its name.

            \begin{42ccode}
int my_func(int arg1, char arg2, char *arg3)
{
    return (my_val);
}

int func2(void)
{
    return ;
}
            \end{42ccode}

        \end{itemize}
        \newpage


%******************************************************************************%
%                        Typedef, struct, enum and union                       %
%******************************************************************************%
    \section{Typedef, struct, enum and union}

        \begin{itemize}

            \item Add a tabulation when declaring a struct, enum or union.

            \item When declaring a variable of type struct, enum or union,
                add a single space in the type.

            \item When declaring a struct, union or enum with a typedef,
                all indentation rules apply. You must align the typedef's name
                with the struct/union/enum's name.

            \item You must indent all structures' names on the same column for their scope.

            \item You cannot declare a structure in a .c file.

        \end{itemize}
        \newpage


%******************************************************************************%
%                                   Headers                                    %
%******************************************************************************%
    \section{Headers}

        \begin{itemize}

            \item The things allowed in header files are:
                header inclusions (system or not), declarations, defines,
                prototypes and macros.

            \item All includes must be at the beginning of the file.

            \item You cannot include a C file.

            \item Header files must be protected from double inclusions. If the file is
            \texttt{ft\_foo.h}, its bystander macro is \texttt{FT\_FOO\_H}.

            \item Unused header inclusions (.h) are forbidden.

            \item All header inclusions must be justified in a .c file
                as well as in a .h file.

        \end{itemize}

        \begin{42ccode}
#ifndef FT_HEADER_H
# define FT_HEADER_H
# include <stdlib.h>
# include <stdio.h>
# define FOO "bar"

int g_variable;
struct s_struct;

#endif
        \end{42ccode}
        \newpage

%******************************************************************************%
%                           Macros and Pre-processors                          %
%******************************************************************************%
    \section{Macros and Pre-processors}

        \begin{itemize}

            \item Preprocessor constants (or \#define) you create must be used
                only for literal and constant values.
            \item All \#define created to bypass the norm and/or obfuscate
                code are forbidden. This part must be checked by a human.
            \item You can use macros available in standard libraries, only
                if those ones are allowed in the scope of the given project.
            \item Multiline macros are forbidden.
            \item Macro names must be all uppercase.
            \item You must indent characters following \#if, \#ifdef
                or \#ifndef.

        \end{itemize}
        \newpage


%******************************************************************************%
%                              Forbidden stuff!                                %
%******************************************************************************%
    \section{Forbidden stuff!}

        \begin{itemize}

            \item You're not allowed to use:

                \begin{itemize}

                    \item for
                    \item do...while
                    \item switch
                    \item case
                    \item goto

                \end{itemize}

            \item Ternary operators such as `?'.

            \item VLAs - Variable Length Arrays.

            \item Implicit type in variable declarations

        \end{itemize}
        \begin{42ccode}
    int main(int argc, char **argv)
    {
        int     i;
        char    string[argc]; // This is a VLA

        i = argc > 5 ? 0 : 1 // Ternary
    }
        \end{42ccode}
        \newpage

%******************************************************************************%
%                                   Comments                                   %
%******************************************************************************%
    \section{Comments}

        \begin{itemize}

            \item Comments cannot be inside functions' bodies.
                Comments must be at the end of a line, or on their own line

            \item You comments must be in English. And they must be
                useful.

            \item A comment cannot justify a "bastard" function.

        \end{itemize}
        \newpage


%******************************************************************************%
%                                    Files                                     %
%******************************************************************************%
    \section{Files}

        \begin{itemize}

            \item You cannot include a .c file.

            \item You cannot have more than 5 function-definitions in a .c file.

        \end{itemize}
        \newpage


%******************************************************************************%
%                                   Makefile                                   %
%******************************************************************************%
    \section{Makefile}

            Makefiles aren't checked by the Norm, and must be checked during evaluation by 
            the student.
            \begin{itemize}

                \item The \$(NAME), clean, fclean, re and all
                  rules are mandatory.

                \item If the makefile relinks, the project will be considered
                  non-functional.

                \item In the case of a multibinary project, in addition to
                  the above rules, you must have a rule that compiles
                  both binaries as well as a specific rule for each
                  binary compiled.

                  \item In the case of a project that calls a function from a non-system library
                  (e.g.: \texttt{libft}), your makefile must compile
                  this library automatically.

                  \item All source files you need to compile your project must
                    be explicitly named in your Makefile.

            \end{itemize}



\end{document}
%******************************************************************************%
