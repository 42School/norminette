\documentclass{42-en}
\newcommand\qdsh{\texttt{42sh}}



%******************************************************************************%
%                                                                              %
%                               Prologue                                       %
%                                                                              %
%******************************************************************************%

\begin{document}
\title{The Norm}
\subtitle{Version 3}

\summary
{
    This document describes the applicable standard (Norm) at 42. A
    programming standard defines a set of rules to follow when writing code.
    The Norm applies to all C projects within the Inner Circle by default, and
    to any project where it's specified.
}

\maketitle

\tableofcontents



%******************************************************************************%
%                                                                              %
%                                 Avant-propos                                 %
%                                                                              %
%******************************************************************************%
\chapter{Foreword}

    This document describes the applicable standard (Norm) at 42. A
    programming standard defines a set of rules to follow when writing code.
    The Norm applies to all C projects within the Inner Circle by default, and
    to any project where it's specified.


    \section{Why impose a standard?}

        The Norm's two main objective :
        1. To format and standardize your code so that anyone (students,
        staff and even yourself) can read and understand them easily.
        2. To guide you in writing short and simple code.

    \section{The Norm for submissions}

        All of the C files of a "normable" project must respect the school's Norm. It will be
        checked during evaluations. If you made any Norm error you'll get
        a 0 for the exercise or even for the whole project. During
        peer-evaluations, your grader will have to launch the "Norminette"
        present in your submission's dumps.

    \section{Suggestions}

        You'll realise soon enough that the Norm isn't as intimidating
        as it seems. On the contrary, it'll help you more than you
        know. It'll allow you to read your peers' code more easily
        and vice versa. A source file containing one Norm error will be
        treated the same way as a source file containing 10 Norm errors.
        We strongly advise you to keep the Norm in mind while coding
        - even though you may feel it's slowing you down at first. In
        time, it'll become a reflex.


    \section{Disclaimers}

        "Norminette" is a program, and all programs are subject to
        bugs. Should you spot one, please report it in the forum's
        appropriate section. However, as the "Norminette" always
        prevails, all your submissions must adapt to its bugs.



%******************************************************************************%
%                                                                              %
%                                 Norme                                        %
%                                                                              %
%******************************************************************************%
\chapter{The Norm}


%******************************************************************************%
%                         Conventions de denomination                          %
%******************************************************************************%
    \section{Denomination}

        \subsubsection{Mandatory part}

            \begin{itemize}

                \item A structure's name must start by
                  \texttt{s\_}.

    			\item A typedef's name must start by
                  \texttt{t\_}.

    			\item A union's name must start by \texttt{u\_}.

    			\item An enum's name must start by \texttt{e\_}.

    			\item A global's name must start by
                  \texttt{g\_}.

    			\item Variables and functions names can
                  only contain lowercases, digits and
                  '\_' (Unix Case).

    			\item Files and directories names can
                  only contain lowercases, digits and
                  '\_' (Unix Case).

    			\item The file must compile.

			    \item Characters that aren't part of the standard
                  ascii table are forbidden.

            \end{itemize}

        \subsubsection{Advice part}

            \begin{itemize}

    			\item Objects (variables, functions, macros, types,
                  files or directories) must have the most
                  explicit or most mnemonic names as possible.
                  Only 'counters' can be named to your liking.

    			\item Abreviations are tolerated as long as it's
                  to shorten the original name, and that it
                  remains intelligible. If the name contains
                  more than one word, words shall be separated
                  by `\_'.

    			\item All identifiers (functions, macros, types,
                  variables, etc) must be in English.

                \item Global variables are forbidden, expect where it's
                  mandatory to use them (Signal handling for one). Using 
                  a global variable in a project where it's not explicitly 
                  allowed is a norm error.

            \end{itemize}


%******************************************************************************%
%                                  Formatage                                   %
%******************************************************************************%
    \section{Formatting}

        \subsubsection{Mandatory part}

            \begin{itemize}

    		        \item You must indent your code with 4-space
                  tabulations. This is not the same as 4 average
                  spaces, we're talking about real tabulations here.

                \item Each function must be maximum 25 lines, not
                  counting the function's own curly brackets.

                \item Each line must be at most 80 columns wide, comments
                  included. Warning : a tabulation doesn't count
                  as a column, but as the number of spaces it
                  represents.

                \item One instruction per line.

          		\item An empty line must be empty: no spaces or tabulations.

      			\item A line can never end with spaces or tabulations.

      			\item You need to start a new line after each curly bracket
                  or end of control structure.

      			\item Unless it's the end of a line, each comma or semi-colon
                  must be followed by a space.

      			\item Each operator (binary or ternary) or operand must be
                  separated by one - and only one - space.

      			\item Each C keyword must be followed by a space, except for
                  keywords for types (such as int, char, float, etc.),
                  as well as sizeof.

      			\item Each variable declaration must be indented on the same
                  column.

      			\item The asterisks that go with pointers must be stuck to
                  variable names.

    		    \item One single variable declaration per line.

      		    \item We cannot stick a declaration and an initialisation
                  on the same line, except for global variables (when allowed) and
                  static variables.

			    \item Declarations must be at the beginning of a function,
                  and must be separated by an empty line.

                \item In a function, you must place an empty line between 
                    variable declarations and the remaining of the function.
                    No other empty lines are allowed in a function.

                \item Multiple assignments are strictly forbidden.

                \item You may add a new line after an instruction or
                  control structure, but you'll have to add an
                  indentation with brackets or affectation operator.
                  Operators must be at the beginning of a line.

            \end{itemize}


%******************************************************************************%
%                            Parametres de fonction                            %
%******************************************************************************%
    \section{Functions parameters}

        \subsubsection{Mandatory part}

            \begin{itemize}

    			\item A function can take 4 named parameters maximum.

			    \item A function that doesn't take arguments must be
                  explicitely pototyped with the word "void" as
                  argument.

            \end{itemize}


%******************************************************************************%
%                                  Fonctions                                   %
%******************************************************************************%
    \section{Functions}

        \subsubsection{Mandatory part}

            \begin{itemize}

          		\item Parameters in functions' prototypes must be named.

      			\item Each function must be separated from the next by
                 an empty line.

                \item You can't declare more than 5 variables per function.

                \item Return of a function has to be between parantheses. 

                \item Each function must have a single tabulation between its
                    return type and its name.

             \end{itemize}


%******************************************************************************%
%                        Typedef, struct, enum et union                        %
%******************************************************************************%
    \section{Typedef, struct, enum and union}

        \subsubsection{Mandatory Part}

            \begin{itemize}

                \item Add a tabulation when declaring a struct, enum or union.

                \item When declaring a variable of type struct, enum or union,
                  add a single space in the type.

      			\item When declaring a struct, union or enum with a typedef,
                        all rules apply. You must align the typedef's name
                        with the struct/union/enum's name.

      			\item You cannot declare a structure in a .c file.

            \end{itemize}


%******************************************************************************%
%                                   Headers                                    %
%******************************************************************************%
    \section{Headers}

        \subsubsection{Mandatory Part}

            \begin{itemize}

                \item The things allowed in header files are :
                  header inclusions (system or not), declarations, defines,
                  prototypes and macros.

      			\item All includes must be at the beginning of the file.

                \item You cannot include a C file.

      			\item We'll protect headers from double inclusions. If the file is
                  \texttt{ft\_foo.h}, its bystander macro is \texttt{FT\_FOO\_H}.

      			\item Unused header inclusions (.h) are forbidden.

                \item All header inclusions must be justified in a .c file
                  as well as in a .h file.

            \end{itemize}


%******************************************************************************%
%                           Macros et pre-processeur                           %
%******************************************************************************%
    \section{Macros and Pre-processors}

        \subsubsection{Mandatory part}

            \begin{itemize}

        	\item Preprocessor constants (or \#define) you create must be used
		  only for associate literal and constant values.
		\item All \#define created to bypass the norm and/or obfuscate
                  code are forbidden. This point must be checked by a human.
		\item You can use macros available in standard libraries, only
		  if those ones are allowed in the scope of the given project.
    		\item Multiline macros are forbidden.
    		\item Only macros names are uppercase.
		\item You must indent characters following \#if , \#ifdef
                  or \#ifndef.

            \end{itemize}


%******************************************************************************%
%                             Choses interdites !                              %
%******************************************************************************%
    \section{Forbidden stuff !}

        \subsubsection{Mandatory part}

            \begin{itemize}

    			\item You're not allowed to use :

                    \begin{itemize}

                      \item for
                	    \item do...while
                	    \item switch
                	    \item case
            		      \item goto

                    \end{itemize}

                \item ternary operators such as `?'.

                \item VLAs - Variable Length Arrays.

            \end{itemize}

%******************************************************************************%
%                                 Commentaires                                 %
%******************************************************************************%
    \section{Comments}

        \subsubsection{Mandatory part}

            \begin{itemize}

    		\item You're allowed to comment your code in your source files.

		\item Comments cannot be inside functions' bodies.

            \end{itemize}

        \subsubsection{Advice part}

            \begin{itemize}

                \item You comments must be in English. And they must be
                  useful.

		\item A comment cannot justify a "bastard" function.

            \end{itemize}


%******************************************************************************%
%                                 Les fichiers                                 %
%******************************************************************************%
    \section{Files}

        \subsubsection{Mandatory part}

            \begin{itemize}

 		\item You cannot include a .c file.

	    	\item You cannot have more than 5 function-definitions in a .c file.

            \end{itemize}


%******************************************************************************%
%                                   Makefile                                   %
%******************************************************************************%
    \section{Makefile}

        \subsubsection{Mandatory part}
            Makefile aren't checked by the Norm, and must be checked during evaluation by 
            the student.
            \begin{itemize}

                \item The \$(NAME), clean, fclean, re and all
                  rules are mandatory.

		        \item If the makefile relinks, the project will be considered
                  non-functional.

      		    \item In the case of a multibinary project, on top of the
                  rules we've seen, you must have a rule that compiles
                  both binaries as well as a specific rule for each
                  binary compiled.

      		    \item In the case of a project that calls a functions library
                  (e.g.: \texttt{libft}), your makefile must compile
                  this library automatically.

      		    \item All source files you need to compile your project must
		            be explicitly named in your Makefile.

            \end{itemize}



\end{document}
%******************************************************************************%
